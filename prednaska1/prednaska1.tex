\documentclass{beamer}
\usetheme[pageofpages=z,
          bullet=circle,
          titleline=true,
          alternativetitlepage=true,
          titlepagelogo=uplogo,
          ]{UVT}

\usepackage[utf8x]{inputenc}
\usepackage{czech}
\usepackage{minted}
\usepackage{amssymb}
\usepackage{amsthm}

\newenvironment{definice}
{
    \begin{center}
    \begin{tabular}{p{9cm}}
    \textbf{Definice:}
}
{
    \end{tabular}
    \end{center}
}


\newenvironment{veta}
{
    \begin{center}
    \begin{tabular}{p{9cm}}
    \textbf{Věta:}
}
{
    \end{tabular}
    \end{center}
}

\newcommand{\sep}{\,|\,}

\newenvironment{itemizex}%
  {\large \begin{itemize}%
    \setlength{\itemsep}{8pt}%
    \setlength{\parskip}{8pt}}%
  {\end{itemize}}

  \newenvironment{itemize4}%
  {\large \begin{itemize}%
    \setlength{\itemsep}{4pt}%
    \setlength{\parskip}{4pt}}%
  {\end{itemize}}

\newenvironment{itemizey}%
  {\large \begin{itemize}%
    \setlength{\itemsep}{6pt}%
    \setlength{\parskip}{6pt}}%
  {\end{itemize}}

\newenvironment{enumeratex}%
  {\large \begin{enumerate}%
    \setlength{\itemsep}{6pt}%
    \setlength{\parskip}{6pt}}%
  {\end{enumerate}}


\title{Teorie informace a kódování (KMI/TIK)}
\subtitle{Bezpečnostní kódy}
\author{Lukáš Havrlant}
\date{13. listopadu 2012}
\institute{Univerzita Palackého}

\begin{document}

\begin{frame}[t,plain]
\titlepage
\end{frame}


\begin{frame}[t,fragile]{Konzultace}
\begin{itemizex}
\item V pracovně 5.076.
\item Každý čtvrtek 9.00--11.00. 
\item Emaily:
  \begin{itemizex}
    \item lukas.havrlant+skola@gmail.com
    \item lukas.havrlant@upol.cz
  \end{itemizex}
\item Web: \url{http://tux.inf.upol.cz/~havrlant/}
\end{itemizex}
\end{frame}


\begin{frame}[t,fragile]{Doporučená literatura}
\begin{itemize}
    \item Adámek J.: Foundations of Coding. J. Wiley, 1991.
    \item Adámek J.: Kódování. SNTL, Praha, 1989. (lze stáhnout na \url{http://notorola.sh.cvut.cz/~bruxy/})
    \item Ash R. B.: Information Theory. Dover, New York, 1990.
\end{itemize}
\end{frame}


\begin{frame}[t,fragile]\frametitle{Úvod} 
    \begin{itemize4}
        \item Při přenosu dat může dojít k chybě $\longrightarrow$ jak odhalit chybu? (znáte: CRC, hash) Jak opravit chybu?
        \item Chyba může být dvojího typu:
            \begin{enumerate}
                \item Záměna znaku: 101010 $\longrightarrow$ 101000 (o tento se budeme zajímat)
                \item Vytvoření nového znaku: 101010 $\longrightarrow$ 1010101. Pohlcení znaku: 101010 $\longrightarrow$ 10110
            \end{enumerate}
        \item Základní princip: redundance informace. 
        \item Čeština je hodně redundantí: \uv{opakkvání} $\longrightarrow$ \uv{opakování}, ale \uv{sekers} $\longrightarrow$ \uv{sekera} nebo \uv{sekery}?
        \item Všechna rodná čísla vytvořená od roku 1986 včetně jsou dělitelná jedenácti. 
    \end{itemize4}
\end{frame}


\begin{frame}[t,fragile]\frametitle{Příklad kódování objevující jednu chybu} 
    \begin{itemizex}
        \item Kontrola sudé parity: kód doplní na konec slova 0/1 tak, aby slovo mělo sudý počet jedniček.
        \item Vstupem je slovo $w$, doplníme $a\in\{0,1\}$, vznikne slovo $wa$.
        \item Pokud $w=001$, pak $a=1$ a $wa=0011$.
        \item Pokud $w=101$, pak $a=0$ a $wa=1010$.
        \item Kód objevuje 1 chybu: pokud přijmeme $1011$, nastala chyba.
        \item Kód nerozpozná dvě chyby! Vyšleme $1010$, přijmeme $1111$. 
    \end{itemizex}
\end{frame}


\begin{frame}[t,fragile]\frametitle{Příklad kódování opravující jednoduché chyby} 
    \begin{itemizex}
        \item Opakující kód -- symbol odešleme třikrát. 
        \item 0 $\longrightarrow$ 000; 1 $\longrightarrow$ 111.
        \item Pokud přijmeme 010, poslali jsme původně 000 (nebo nastala více než jedna chyba).
    \end{itemizex}
\end{frame}


\begin{frame}[t,fragile]\frametitle{Cíl} 
    \begin{itemizex}
        % \item Opakující kód je neefektivní, pokud máme slovo délky $n$, po zakódování má délku $3n$.
        \item Cíl: sestrojit kód, který objeví/opraví co nejvíce chyb při co nejnižší redundanci.
        \item Uvažujeme pouze blokové kódy.
    \end{itemizex}
\end{frame}



\begin{frame}[t,fragile]\frametitle{Kódová a nekódová slova} 
    \begin{itemizey}
        \item Zdrojová abeceda $A$, kódová abeceda $B$, délka kódového slova je $n$, pak pro kód $C$ platí: $C\subseteq B^n$.
        \item $B^n=\{b_1b_2\dots b_n|b_i\in B \mbox{ pro } i = 1,2\dots n\}$
        \item $C$ je množina kódových slov, $B^n\setminus C$ množina nekódových slov.
        \item Přijmeme-li nekódové slovo, tak během přenosu nastala chyba.
        \item Pokud přijmeme kódové slovo, tak během přenosu\dots?
        \item Kontrola sudé parity: pro $n=4$ máme 
            \begin{itemize}
                \item $B^n=\{0000, 0001, 0010, 0011, 0100, 0101, \dots\}; \quad |B^n|=2^4=16$
                \item $C=\{0000, 0011, 0101, 0110, \dots\}; \quad |C|=8$
            \end{itemize}
    \end{itemizey}
\end{frame}



\begin{frame}[t,fragile]\frametitle{Informační poměr kódu (rate)} 
    \begin{itemize4}
        % \item Typický způsob kódování slova:
        \item Máme slovo délky $k$. Přidáme symboly z kódové abecedy a získáme slovo o délce $n$.
        \item Původním $k$ symbolům říkáme \textit{informační symboly}.
        \item $n-k$ symbolům říkáme \textit{kontrolní symboly}.
        \item Poměr $R(C)=\frac{k}{n}$ měří redundanci kódu.
    \end{itemize4}

    \begin{definice}
    Nechť $C$ je blokový kód o délce $n$. Pokud $|C|=|B|^k$ pro nějaké $k\in\mathbb{N}$, řekneme, že \textit{informační poměr kódu} $R(C)$ je
    $$
    R(C) = \frac{k}{n}.
    $$
    \end{definice}
\end{frame}


\begin{frame}[t,fragile]\frametitle{Příklady informačních poměrů} 
    \begin{itemizex}
        \item \textbf{Opakovací kód}
        \begin{itemize4}
            \item Kód $C_n$ má na abecedě $B$ kódová slova $\{a^n\sep a\in B\}$
            \item Pro $B=\{0,1\}$ máme $C_3=\{000,111\}, C_5=\{00000,11111\}$
            \item $|B|=|C|, k=1;\quad R(C_n)=\frac{1}{n}$
        \end{itemize4}
        \item \textbf{Kontrola sudé parity}
        \begin{itemize4}
            \item Pro kód délky $n$ platí: $C_n=\{w \mbox{ má sudý počet 1}\sep w\in \{0,1\}^n\}$
            \item $C_3=\{000,011,101,110\}$
            \item Pro $C_3$: $|B|=2, |C|=4, k=2;\quad R(C_3)=\frac23$
            \item Pro $C_n$: $|B|=2, |C|=2^{n-1}, k=n-1;\quad R(C_n)=\frac{n-1}{n}$
        \end{itemize4}
    \end{itemizex}
\end{frame}


\begin{frame}[t,fragile]\frametitle{Hammingova vzdálenost} 
    \begin{definice} Pro slova $u=u_1\cdots u_n, v=v_1\cdots v_n\in B^n$ definujeme zobrazení
    $$
    \delta(u,v) = |\{i\sep 1\le i \le n, u_i\ne v_i\}|.
    $$
    Tj. $\delta(u,v)$ vrací počet pozic, na kterých se slova $u$ a $v$ liší. Zobrazení $\delta$ nazýváme \textbf{Hammingova vzdálenost}. $\delta$ je metrikou na $B^n$:
    \begin{eqnarray*}
    \delta(u,v)&\ge&0\quad (=0 \mbox{ iff } u=v),\\
    \delta(u,v)&=&\delta(v,u),\\
    \delta(u,v)&\le&\delta(u,w)+\delta(w,v).
    \end{eqnarray*}

    $\delta(011,101)=2, \quad \delta(011, 111)=1,\quad \delta(0110, 1001)=4,$ $\delta(0123, 0145)=2, \quad\delta(01,01)=0$
    \end{definice}
\end{frame}


\begin{frame}[t,fragile]\frametitle{Důkaz trojúhelníkové nerovnosti} 
Chceme dokázat: $\delta(u,v)\le\delta(u,w)+\delta(w,v)$. Označme:
\begin{eqnarray*}
U&=&\{i\sep 1\le i \le n, u_i\ne w_i\},\\
V&=&\{i\sep 1\le i \le n, w_i\ne v_i\}.
\end{eqnarray*}
Pak $U\cup V$ je množina indexů, na kterých se $u$ může lišit od $v$. Pro všechna $i\in \{1,2,\dots,n\}\setminus(U\cup V)$ platí, že $u_i=w_i=v_i$. Dále
\begin{eqnarray*}
\delta(u,v)&\le&|U\cup V|.
\end{eqnarray*}
$\delta(u,v)$ může být menší, pokud $u_i\ne w_i \wedge w_i\ne v_i$, ale $u_i=v_i$. Pak $i\in U\cup V$, ale $i\notin \{j\sep 1\le j \le n, u_j\ne v_j\}$. Dostáváme:
\begin{equation*}
    \delta(u,v)\le|U\cup V|\le|U|+|V|=\delta(u,w)+\delta(w,v).\qed
\end{equation*}
\end{frame}


\begin{frame}[t,fragile]\frametitle{Minimální vzdálenost} 
\begin{definice}
Minimální vzdálenost kódu $C$, zapíšeme $d(C)$, je minimální vzdálenost dvou různých slov z $C$:
$$
d(C) = \min\{\delta(u,v)\sep u,v\in C, u\ne v\}.
$$
\end{definice}

Příklady:
\begin{eqnarray*}
C_1=\{0011, 1010, 1111\},& d(C_1)=2\\ 
C_2=\{0011, 1010, 0101, 1111\},& d(C_2)=2\\ 
C_3=\{000, 111, 100\},& d(C_3)=1
\end{eqnarray*}
\end{frame}


\begin{frame}[t,fragile]\frametitle{Koule} 
\begin{definice}
Nechť $C$ je nějaký kód o délce $n$ a $u\in C$. Pak $B_q(u)=\{v\in B^n\sep \delta(u,v)\le q\}$ nazveme koule se středem v $u$ a poloměrem $q$.
\end{definice}

Důsledek: Kód $C$ má minimální vzdálenost $d(C)\ge d$ právě tehdy, když pro všechna $u\in C$ platí
$$
B_{d-1}(u) \cap C = \{u\}.
$$
\end{frame}


\begin{frame}[t,fragile]\frametitle{Objevování chyb} 
    \begin{itemizex}
        \item $t$ chyb nastane, pokud pošleme slovo $u$, přijmeme slovo $v$ a $\delta(u,v)=t$. (Slovo $u$ se od $v$ liší v $t$ symbolech.)
    \end{itemizex}

\begin{definice}
Kód $C$ \textit{objevuje} $d$ chyb, pokud při odeslání kódového slova a vzniku $t\le d$ chyb, přijmeme vždy nekódové slovo.
\end{definice}

\begin{veta}
Kód $C$ \textit{objevuje} $d$ chyb právě když $d(C)>d$.
\end{veta}

Největší $d$ takové, že $C$ objevuje $d$ chyb je $d(C)-1$.
\end{frame}


\begin{frame}[t,fragile]\frametitle{Příklady} 
    \begin{itemizex}
        \item \textbf{Kontrola sudé parity:}
        \begin{itemize}
            \item $d(C)=2$, takže kód objevuje 1 chybu.
            \item Proč $d(C)=2$? Nechť $u,v\in B^{n-1}, u\ne v$. Slova zakódujeme, dostaneme $ua, vb$, kde $a, b\in B$. 
            \begin{itemize}
                \item Pokud $\delta(u,v)\ge2$, pak jistě $\delta(ua, vb)\ge2$.
                \item Pokud $\delta(u,v)=1$, pak právě jedno ze slov obsahuje lichý počet jedniček a tedy $a\ne b$ a $\delta(ua, vb)=2$. 
            \end{itemize}
            % \item Kód neopravuje žádnou chybu.
        \end{itemize}
        \item \textbf{Opakovací kód:}
            \begin{itemize}
                \item $d(C_n)=n$, takže kód objevuje $n-1$ chyb.
                % \item Kód může opravovat nějaké chyby.
            \end{itemize}
    \end{itemizex}
\end{frame}


\begin{frame}[t,fragile]\frametitle{Oprava chyb} 
    \begin{itemizex}
        \item Pokud přijmeme nekódové slovo $v$, \textit{opravíme} ho na takové kódové slovo $u$, které je slovu $v$ nejblíže. 
        \item $\longrightarrow$ pro každé nekódové slovo musí existovat jediné kódové slovo, které je mu nejblíže. 
    \end{itemizex}

\begin{definice}
Kód $C$ opravuje $d$ chyb pokud pro každé kódové slovo $u\in C$ a každé slovo $v\in B^n$ takové, že $\delta(u, v) \le d$, platí, že slovo $u$ má od slova $v$ nejmenší vzdálenost ze všech slov z $C$. \\Tedy pokud $u\in C$ a $\delta(u,v)\le d$, pak $\delta(u,v)<\delta(w,v)$ pro všechna $w\in C, w\ne u$.
\end{definice}
\end{frame}


\begin{frame}[t,fragile]\frametitle{Oprava chyb podruhé} 
\begin{veta}
$C$ opravuje $d$ chyb právě tehdy, když $d(C)>2d$.
\end{veta}

\begin{itemizex}
    \item Největší $d$ takové, že $C$ opravuje $d$ chyb je $d=\lfloor(d(C)-1)/2\rfloor$. Příklady:
    \item Kontrola sudé parity má $d(C)=2$, takže opravuje 0 chyb. 
    \item Opakovací kód $C_3$ opravuje $(3-1)/2=1$ chybu.
\end{itemizex}
\end{frame}



\begin{frame}[t,fragile]\frametitle{Důkaz} 
\begin{veta}$C$ opravuje $d$ chyb právě tehdy, když $d(C)>2d$.\end{veta}
\uv{$\Leftarrow$}: Nechť $d(C)>2d$. Dále nechť $u\in C, v\in B^n:$ $\delta(u,v)\le d$. Sporem předpokládejme, že pro $w\in C, w\ne u$ platí $\delta(w,v)\le\delta(u,v)$. Použijeme trojúhelníkovou nerovnost:
$$
\delta(w,v)+\delta(u,v)\ge\delta(w,u)
$$
Přitom ale $\delta(u,v)\le d$ a zároveň $\delta(w,v)\le d$, takže:
$$
2d=d+d\ge\delta(w,v)+\delta(u,v)\ge\delta(w,u).
$$
Dostáváme $2d\ge\delta(w,u)$, což je spor s předpokladem, protože $u,w\in C$.
\end{frame}


\begin{frame}[t,fragile]\frametitle{Důkaz: pokračování} 
\begin{veta}$C$ opravuje $d$ chyb právě tehdy, když $d(C)>2d$.\end{veta}
\uv{$\Rightarrow$}: Nechť $C$ opravuje $d$ chyb. Opět sporem předpokládejme, že $d(C)\le 2d$ a že tak existují $u,w\in C$ taková, že $t=\delta(u,w)\le2d$. Rozdělme důkaz na dva případy.

Když je $t$ sudé. Dále nechť $v$ je slovo, které má vzdálenost $t/2$ od obou slov $u, w$. Slovo $v$ sestrojíme takto: nechť $I=\{i\sep u_i\ne w_i\}$, takže $|I|=t$. Dále $J\subset I$ a $|J|=t/2$. Takže pokud $I=\{i_1, i_2, \dots, i_t\}$, tak $J$ může např. vypadat $J=\{i_1, i_2, \dots, i_{t/2}\}$. Pak $v_i=u_i$ pro všechna $i\in J$ a $v_i=w_i$ pro všechna $i\notin J$.

Potom $\delta(u,v)=t/2=\delta(u,w)$. Proto $\delta(u,v)=t/2\le d$, ale $\delta(u,v)=\delta(u,w)$, což je spor s tím, že $C$ opravuje $d$ chyb. 
\end{frame}


\begin{frame}[t,fragile]\frametitle{Důkaz: finále} 
Když je $t$ liché. Pak $t+1$ je sudé, přitom $t+1\le2d$. Stejně jako v předchozím kroku, nechť $v$ má vzdálenost $(t+1)/2$ od obou slov $u$ a $w$. Tedy $\delta(u,v)=(t+1)/2\le d$, ale $\delta(u,v)=\delta(u,w)$, což je spor s předpokladem, že kód $C$ opravuje $d$ chyb.\qed
\end{frame}


\begin{frame}[t,fragile]\frametitle{Příklady} 
    \begin{itemizex}
        \item \textbf{Kontrola sudé parity:} $d(C)=2$, kód neopravuje žádnou chybu.
        \item \textbf{Opakovací kód:} $d(C_n)=n$, kód opravuje $\lfloor(n-1)/2\rfloor$ chyb. 
    \end{itemizex}


\end{frame}



\begin{frame}[t,fragile]\frametitle{Dvoudimenzionální kontrola sudé parity} 
Informační symboly umístíme do matice $p-1\times q-1$, ke které přidáme jeden sloupec a jeden řádek pro kontrolní součty, tj. aby celý řádek/sloupec obsahoval sudý počet jedniček. Poslední bit nastavíme tak, aby celé slovo mělo sudou paritu. Příklad pro $(p,q)=(4,5)$:

$$
\begin{array}{cccc|c}
0&1&1&0&0\\
0&0&1&0&1\\
1&1&1&0&1\\\hline
1&0&1&0&0
\end{array}
$$

Kód opravuje 1 chybu. 
\end{frame}


\begin{frame}[t,fragile]\frametitle{Kód dva-z-pěti} 
    \begin{itemize4}
        \item Binární blokový kód s délkou $n=5$.
        \item Kódová slova obsahují 2 jedničky. Celkem: ${5\choose2}=10$ možností.
        \item $00011, 00101, 00110,01001,01010,\dots$
        \item Populární na reprezentování číslic. 
        \item Dekódování: Můžeme přidat váhu jednotlivým pozicím: 0, 1, 2, 3, 6. Potom můžeme kódové slovo $00101$ dekódovat jako: $0\cdot0+0\cdot1+1\cdot2+0\cdot3+1\cdot6=8$.
        \item Trojka je reprezentována dvěma slovy: $10010$ a $01100$. Druhá možnost se používá pro nulu. 
        % \item $d(C)=2$, informační poměr kódu nemůžeme určit, $|C|=10$ není mocnina dvou.
    \end{itemize4}
\end{frame}


\begin{frame}[t,fragile]\frametitle{Systematický kód} 
\begin{definice}
Blokový kód $C$ o délce $n$ nazýváme \textbf{systematický} kód, pokud pro nějaké $k<n$, pro všechna slova $u\in B^k$, existuje právě jedno slovo $v\in B^{n-k}$ takové, že $uv\in C$ je kódové slovo.
\end{definice}

\begin{itemizex}
    \item Opakovací kód a kód sudé parity jsou systematické kódy.
    \item $R(C)=\frac{k}{n}$
\end{itemizex}
\end{frame}

\begin{frame}[t,fragile]\frametitle{Cvičení} 
    \begin{itemizex}
        \item Jaká je $d(C)$ kódu dva-z-pěti? Jaký je informační poměr?
        \item Jaká je $d(C)$ dvoudimenzionální kontroly sudé parity? Jaký je informační poměr?
        \item Jaká je největší možná $d(C)$ systematického kódu, který má délku $n$ a $k$ informačních znaků?
        \item Za použití dvoudimenzionálního $(4,5)$-kódu (4 řádky, 5 sloupců) jsme přijali slovo 01001 11101 11100 01100. Je to kódové slovo? Pokud ne, lze opravit? 
    \end{itemizex}
\end{frame}

\end{document}