\documentclass{beamer}
\usetheme[pageofpages=z,
          bullet=circle,
          titleline=true,
          alternativetitlepage=true,
          titlepagelogo=uplogo,
          ]{UVT}

\usepackage[utf8x]{inputenc}
\usepackage{czech}
\usepackage{minted}
\usepackage{amssymb}
\usepackage{amsthm}

\newenvironment{definice}
{
    \begin{center}
    \begin{tabular}{p{9cm}}
    \textbf{Definice:}
}
{
    \end{tabular}
    \end{center}
}

\newenvironment{lema}
{
    \begin{center}
    \begin{tabular}{p{9cm}}
    \textbf{Lemma:}
}
{
    \end{tabular}
    \end{center}
}


\newenvironment{veta}
{
    \begin{center}
    \begin{tabular}{p{9cm}}
    \textbf{Věta:}
}
{
    \end{tabular}
    \end{center}
}

\newcommand{\sep}{\,|\,}
\newcommand{\zero}{\textbf{0}}
\newcommand{\emptyline}{\\$\,$\\}

\newenvironment{itemizex}%
  {\large \begin{itemize}%
    \setlength{\itemsep}{8pt}%
    \setlength{\parskip}{8pt}}%
  {\end{itemize}}

  \newenvironment{itemize4}%
  {\large \begin{itemize}%
    \setlength{\itemsep}{4pt}%
    \setlength{\parskip}{4pt}}%
  {\end{itemize}}

\newenvironment{itemizey}%
  {\large \begin{itemize}%
    \setlength{\itemsep}{6pt}%
    \setlength{\parskip}{6pt}}%
  {\end{itemize}}

\newenvironment{enumeratex}%
  {\large \begin{enumerate}%
    \setlength{\itemsep}{6pt}%
    \setlength{\parskip}{6pt}}%
  {\end{enumerate}}


\title{Teorie informace a kódování (KMI/TIK)}
\subtitle{Binární lineární kódy}
\author{Lukáš Havrlant}
\date{20. listopadu 2012}
\institute{Univerzita Palackého}

\begin{document}

\begin{frame}[t,plain]
\titlepage
\end{frame}



\begin{frame}[t,fragile]\frametitle{Lineární kódy} 
    \begin{itemizex}
        \item Základní myšlenka: ke kódové abecedě $B$ přidáme algebraické operace $+$ a $\cdot$ tak, abychom z $B$ dostali těleso a z $B^n$ vektorový prostor.
        \item Kódy potom můžeme popsat rovnicemi. 
    \end{itemizex}
\end{frame}


\begin{frame}[t,fragile]\frametitle{Binární lineární kódy na $Z_2=\{0,1\}$}
$$
\begin{array}{c|cc}
+&0&1\\\hline
0&0&1\\
1&1&0
\end{array}
\quad\mbox{a}\quad
\begin{array}{c|cc}
\cdot&0&1\\\hline
0&0&0\\
1&0&1
\end{array}
$$

\begin{itemizex}
    \item 0 je neutrální vzhledem k $+$, máme tak inverzní prvky vzhledem k $+$: $0+0=0$, takže $0=-0$ a $1+1=0$, takže $1=-1$.
    \item Slova můžeme vidět jako vektory: $1101 = \left<1,1,0,1\right>$
    \item Skalární násobení: $a\cdot\left<u_1, \dots, u_n\right> = \left<a\cdot u_1, \dots, a\cdot u_n\right>, a\in Z_2$.
    \item Sčítání: $\left<u_1, \dots, u_n\right> + \left<v_1, \dots, v_n\right> = \left<u_1+v_1, \dots, u_n+v_n\right>$.
\end{itemizex}
\end{frame}



\begin{frame}[t,fragile]\frametitle{Kódy jako rovnice} 
    \begin{itemizex}
        \item Důležité kódy jsme schopni popsat rovnicemi:
        \item Pro slovo $x$ délky $n$ u kódu kontroly sudé parity
$$
x_1+x_2+\dots+x_n=0
$$
        \item Řešením této rovnice jsou všechna kódová slova.
        \item Opakovací kód $C_n = \{a^n\sep a\in Z_2\}$, např. $C_3=\{000, 111\}$.
  \begin{eqnarray*}
  x_1+x_n&=&0\\
  \dots\\
  x_{n-1}+x_n&=&0
  \end{eqnarray*}
    \end{itemizex}
\end{frame}


\begin{frame}[t,fragile]\frametitle{Vektorový podprostor} 
    \begin{itemizex}
        \item Opakování: nechť $V$ je vektorový prostor na množině $K$. Pak $C\subset V$ je vektorový podprostor, pokud:
      \begin{itemizex}
        \item $\forall u,v \in C:\quad u+v\in C$
        \item $\forall a\in K, u \in C:\quad a\cdot u\in C$
      \end{itemizex}
      \item Množina všech řešení systému homogenních rovnic tvoří vektorový podprostor.
      \item Tj. $Z_2^n$ tvoří vektorový prostor, množina řešení $C$ tvoří vektorový podprostor.
    \end{itemizex}
\end{frame}



\begin{frame}[t,fragile]\frametitle{Binární lineární blokové kódy} 
    \begin{definice}
Binární blokový kód $C$ se nazývá lineární, pokud $\forall u,v\in C$ platí $u+v=C$.
    \end{definice}

    \begin{itemizex}
      
      \item V našem případě: $V=\{0,1\}^n, K=\{0,1\}$
      \item Proč tam nemusí být podmínka $\forall a\in K, u \in C:\quad a\cdot u\in C$?
      \item Protože $1\cdot u = u$ a $0\cdot u = \textbf{0}$ $\longrightarrow$ potřebujeme, aby $C$ obsahovalo nulový vektor. Ten obsahuje, protože $u+u=\textbf{0}$.
    \end{itemizex}
\end{frame}


\begin{frame}[t,fragile]\frametitle{} 
  \begin{lema}
    Pro $n\in\mathbb{N}$, množina $C$ všech řešení homogenního systému rovnic na množině $Z_2$ je vektorový podprostor $Z_2^n$. Tedy $C$ je lineární blokový kód.
  \end{lema}

  \begin{itemizex}
    \item Důsledek: pokud popíšeme blokový kód homogenním systémem rovnic, je tento kód lineární.
    \item Je kód dva-z-pěti lineární? (Pro připomenutí: délka 5, obsahuje vždy dvě 1: 00011, 01010, atd.)
  \end{itemizex}
\end{frame}


\begin{frame}[t,fragile]\frametitle{Chybové slovo, Hammingova váha} 
    \begin{definice}
        Pokud pošleme slovo $u$ a přijmeme slovo $v$, pak slovo $e$, které má $1$ na pozicích, na kterých se $u$ od $v$ liší, se nazývá \textit{chybové slovo}.
    \end{definice}

    Přitom platí: $v = u + e$ a také $e = v-u\, (= v+u)$.

    \begin{definice}
    \textit{Hammingova váha} slova $u$ je rovna počtu symbolů v $u$, které se liší od $0$.
    \end{definice}

    \begin{definice}
    \textit{Minimální váha} netriviálního kódu $C$ je nejmenší Hammingova váha slova z $C$ kromě nulového slova.
    \end{definice}
\end{frame}



\begin{frame}[t,fragile]\frametitle{Vztah minimální váhy a minimální vzdálenosti kódu} 
    \begin{veta}
    Minimální váha netriviálního kódu $C$ je rovna $d(C)$.
    \end{veta}

    Důkaz: Směr $\mbox{min\_weight(C)}\ge d(C)$: Nechť $c$ je minimální váha $C$. Nechť $u$ je slovo s váhou $c$. Zřejmě $c=\delta(u, 0\dots0)\ge d(C)$.
\emptyline
    Směr $\mbox{min\_weight(C)}\le d(C)$: Nechť $d(C)=\delta(u,v)$ pro nějaká $u,v\in C$. Pak jistě také $u+v\in C$. Přitom Hammingova váha $u+v$ je rovna $\delta(u,v)$, protože $u+v$ má $1$ na těch pozicích, na kterých se slova $u,v$ liší. Tedy $\mbox{min\_weight(C)}\le d(C)$.
\emptyline
    Celkově tak máme $d(C) = \mbox{min\_weight(C)}$.
\end{frame}


\begin{frame}[t,fragile]\frametitle{Kontrolní matice kódu} 
    \begin{definice}
        Kontrolní matice binárního blokového kódu $C$ o délce $n$ je binární matice $H$ taková, že pro všechny $x=(x_1,\dots, x_n)\in\{0,1\}^n$ máme
$$
H
\begin{pmatrix}
x_1\\
x_2\\
\vdots\\
x_n
\end{pmatrix}
=
\begin{pmatrix}
0\\
0\\
\vdots\\
0
\end{pmatrix}
$$
    \end{definice}

    Matice $H$ je kontrolní maticí kódu $C$, pokud platí:

$$
C=\{x\in\{0,1\}^n\sep Hx^T=\textbf{0}^T\}
$$
\end{frame}


\begin{frame}[t,fragile]\frametitle{Příklady kontrolních matic} 
    \begin{itemizey}
        \item Pokud jsou řádky kontrolní matice nezávislé, $H$ je $n-k\times n$ matice. Kód pak má $n-k$ kontrolních symbolů a $n$ celkových.
        % \item $n-k$ je počet kontrolních symbolů = počet řádků matice.
        \item Kontrolní matice kódu kontroly sudé parity o délce $n$ je matice $1\times n$:
$$
H = (1\dots1)
$$
        \item Kontrolní matice pro opakovací kód délky $5$ je matice $4\times5$:
$$
H=
\begin{pmatrix}
1&1&0&0&0\\
1&0&1&0&0\\
1&0&0&1&0\\
1&0&0&0&1
\end{pmatrix}
$$
    \end{itemizey}
\end{frame}


\begin{frame}[t,fragile]\frametitle{Program, který vykreslí čtverec o dané délce hrany} 
\begin{minted}[linenos=true]{c} 
  printf("*******\n");
}
if(e==8){
  printf("********\n");
  printf("********\n");
  printf("********\n");
  printf("********\n");
  printf("********\n");
  printf("********\n");
  printf("********\n");
  printf("********\n");
}
if(e==9){
  printf("*********\n");
\end{minted}
\end{frame}




\begin{frame}[t,fragile]\frametitle{Věta o kontrolní matici I} 
    \begin{veta}
Pokud binární kód $C$ opravuje jednoduché chyby, pak kontrolní matice kódu $C$ má nenulové a navzájem různé sloupce. 
    \end{veta}

    Důkaz: Pokud $C$ opravuje jednoduché chyby, pak $d(C)>2$ a minimální váha je také $>2$. Nechť $H$ je kontrolní matice pro $C$. Nechť vektor $b^i\in\{0,1\}^n$ má $1$ právě na pozici $i$. Nechť vektor $b^{i,j}\in\{0,1\}^n$ má $1$ právě na pozicích $i$ a $j$. Pokud by matice $H$ měla $i$-tý sloupec nulový, pak $Hb^{i^T}=\zero^T$. Ale $b^i$ je přitom slovo s Hammingovou váhou $1$, což je spor s tím, že minimální váha kódu $C$ je $>2$.
\emptyline
    Pokud by $H$ měla stejné sloupce $i$ a $j$, pak $Hb^{{i,j}^T}=\zero^T$. ($Hb^{{i,j}^T}$ je součet sloupců $i$ a $j$). Nicméně $b^{i,j}$ je slovo s váhou $2$, což je opět spor s tím, že minimální vzdálenost kódu $C$ je $>2$. $\qed$
\end{frame}


\begin{frame}[t,fragile]\frametitle{Věta o kontrolní matici II} 
    \begin{veta}
    Každá binární matice s nenulovými a navzájem různými sloupci je kontrolní matice binárního lineárního kódu, který opravuje jednoduché chyby.
    \end{veta}

    Důkaz: Nechť $H$ je matice, která má nenulové a navzájem různé sloupce. Z předchozího slajdu je zřejmé, že žádné slovo $b^i$ a $b^{i, j}$ není kódovým slovem, tedy minimální váha a tím i minimální vzdálenost kódu je $>2$. $\qed$
\end{frame}



\begin{frame}[t,fragile]\frametitle{Příklad} 
$$
H=\begin{pmatrix}
0&0\\
0&1\\
1&1
\end{pmatrix}
$$
\dots je kontrolní matice binárního lineárního kódu. Přepíšeme do rovnic:
\begin{eqnarray*}
(0x_1+0x_2&=&0)\\
x_2&=&0\\
x_1+x_2&=&0
\end{eqnarray*}

Systém má jediné, nulové řešení. Systém tak generuje kód $C=\{00\}$.
\end{frame}


\begin{frame}[t,fragile]\frametitle{Přidáme sloupec\dots} 
$$
H=\begin{pmatrix}
0&0&1\\
0&1&0\\
1&1&0
\end{pmatrix}
$$

Jak vypadá $C$?
\end{frame}


\begin{frame}[t,fragile]\frametitle{Přidáme ještě jeden sloupec\dots} 
$$
H=\begin{pmatrix}
0&0&1&0\\
0&1&0&1\\
1&1&0&0
\end{pmatrix}
$$

\begin{itemizey}
    \item Kódová slova mají délku 4. Jak vypadá $C$? 
    \item Jak by vypadalo $C$, kdybychom prohodili poslední dva sloupce?
    \item Přidáním dalších sloupců zvyšujeme počet informačních symbolů.
    \item Kolik sloupců může matice nejvýše mít?
    \item Takový kód opravuje právě jednu chybu. Proč ne dvě?
\end{itemizey}
\end{frame}


\begin{frame}[t,fragile]\frametitle{Kód opravuje jednu chybu} 
Aby opravoval dvě chyby, musel by mít kód $C$ minimální váhu $>4$. Tedy matice musí mít $>4$ sloupce. Ovšem pokud má 3 řádky, pak jistě existují tři nebo čtyři sloupce, jejichž součet je nulový sloupec.
\emptyline
Vezmeme libovolné čtyři sloupce $i, j, k, l$. Protože matice má 3 řádky, hodnost je maximálně 3, tak určitě alespoň jeden ze sloupců je lineárně závislý. Nechť $i$-tý sloupec je lineárně závislý, tj. $H_i=a_j\cdot H_j+a_k\cdot H_k+a_l\cdot H_l$. Dva nebo tři koeficienty musí být rovné 1 -- řekněme, že buď $a_j, a_k$, nebo $a_j, a_k, a_l$. Pak součet sloupců $i,j,k$ nebo sloupců $i,j,k,l$ je roven nulovému sloupci.
\emptyline
Což znamená, že pro vektor $u$ s $1$ na pozicích $i,j,k$ nebo $i,j,k,l$ platí $Hu^T=\zero^T$, tedy $u$ je kódové slovo. Ale váha $u$ je tři, nebo čtyři, stejně tak minimální váha (a tím pádem i vzdálenost) kódu. 
\end{frame}



\begin{frame}[t,fragile]\frametitle{Hammingovy kódy} 
    \begin{itemizex}
        \item Perfektní kódy pro opravu jednoduchých chyb.
        \item Kódová slova mají délku $n=2^m-1$, $m$ symbolů je kontrolních. Minimální vzdálenost je 3.
        \item Nazývají se $(2^m-1,2^m-m-1)$ kódy, např. $(7,4)$.
        \item Sloupce kontrolní matice jsou všechny nenulové vektory seřazené lexikograficky (= podle hodnot, pokud bychom je převedli na číslo v desítkové soustavě). Pro $m=2$:
$$
H=
\begin{pmatrix}
0&1&1\\
1&0&1
\end{pmatrix}
$$
    \end{itemizex}
\end{frame}


\begin{frame}[t,fragile]\frametitle{Definice Hammingova kódu} 
    \begin{definice}
Hammingův kód je binární lineární kód, který má, pro nějaké $m$, $m\times2^m-1$ kontrolní matici, jejíž sloupce obsahují všechny nenulové binární vektory. 
    \end{definice}

    Příklad kontrolní matice (jedné z mnoha) pro $m=3$. Dostáváme matici $3\times7$:

$$H=
\begin{pmatrix}
0&0&0&1&1&1&1\\
0&1&1&0&0&1&1\\
1&0&1&0&1&0&1
\end{pmatrix}$$
\end{frame}


\begin{frame}[t,fragile]\frametitle{Zakódování} 
Předchozí matici můžeme přepsat na tento systém rovnic:

$$
\begin{array}{ccccccccc}
&&&x_4+&x_5+&x_6+&x_7&=&0\\
&x_2+&x_3+&&&x_6+&x_7&=&0\\
x_1+&&x_3+&&x_5+&&x_7&=&0
\end{array}
$$
A to můžeme přepsat na:
\begin{eqnarray*}
x_5&=&x_2+x_3+x_4\\
x_6&=&x_1+x_3+x_4\\
x_7&=&x_1+x_2+x_4
\end{eqnarray*}

Vidíme, že $x_1, x_2, x_3, x_4$ můžeme brát jako informační symboly a $x_5, x_6, x_7$ jako kontrolní, které dopočítáme. 
\end{frame}



\begin{frame}[t,fragile]\frametitle{Dekódování} 
    \begin{itemizex}
        \item Pokud je $u$ kódové slovo, pak $Hu^T=\zero^T$. 
        \item Pokud nastane chyba na pozici $i$, přijmeme slovo $v=u+e^i$.
$$
Hv^T=H(u+e^i)^T=Hu^T+He^{i^T}=\zero^T+He^{i^T}=He^{i^T}=H_i
$$
        \item Součin $Hv^T$ je tak rovný $i$-tému sloupci matice $H$.
        \item V $i$-tém sloupci je uloženo číslo $i$ v binární podobě.
        \item $\longrightarrow$ opravíme $i$-tý znak ve slově $v$.
    \end{itemizex}
\end{frame}



\begin{frame}[t,fragile]\frametitle{Informační poměr} 
    \begin{itemizey}
        \item Informační poměr Hammingova kódu je $R(C)=\frac{2^m-m-1}{2^m-1}=$\\$=1-\frac{m}{2^m-1}$ $\longrightarrow$ malá redundance pro velká $m$.
        \item Vždy jsme ale schopni opravit jen jednu chybu.
        \item Hammingovy kódy jsou perfektní, protože mají nejmenší možnou redundanci na množině kódů opravující jednoduché chyby.
        \item Každé slovo délky $n=2^m-1$ je buď kódové, nebo je od kódového slova vzdáleno o jedna.
        \item (Během dekódování buď získáme kódové slovo, nebo nám stačí změnit jeden bit.)
    \end{itemizey}
\end{frame}

\end{document}