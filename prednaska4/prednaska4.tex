\documentclass{beamer}
\usetheme[pageofpages=z,
          bullet=circle,
          titleline=true,
          alternativetitlepage=true,
          titlepagelogo=uplogo,
          ]{UVT}

\usepackage[utf8x]{inputenc}
\usepackage{czech}
\usepackage{minted}
\usepackage{amssymb}
\usepackage{amsthm}

\newenvironment{definice}
{
    \begin{center}
    \begin{tabular}{p{9cm}}
    \textbf{Definice:}
}
{
    \end{tabular}
    \end{center}
}


\newenvironment{veta}
{
    \begin{center}
    \begin{tabular}{p{9cm}}
    \textbf{Věta:}
}
{
    \end{tabular}
    \end{center}
}

\newcommand{\sep}{\,|\,}
\newcommand{\G}{\textbf{G}}
\newcommand{\F}{\textbf{F}}
\newcommand{\vu}{\textbf{u}}
\newcommand{\vv}{\textbf{v}}
\newcommand{\e}{\textbf{e}}
\newcommand{\s}{\textbf{s}}
\newcommand{\vr}{\textbf{r}}
\newcommand{\cela}{\mathbb{Z}}
\renewcommand{\span}{\mbox{span}}
\newcommand{\zero}{\textbf{0}}
\newcommand{\one}{\textbf{1}}
\newcommand{\emptyline}{\\$\,$\\}

\newenvironment{itemizex}%
  {\large \begin{itemize}%
    \setlength{\itemsep}{8pt}%
    \setlength{\parskip}{8pt}}%
  {\end{itemize}}

  \newenvironment{itemize4}%
  {\large \begin{itemize}%
    \setlength{\itemsep}{4pt}%
    \setlength{\parskip}{4pt}}%
  {\end{itemize}}

\newenvironment{itemizey}%
  {\large \begin{itemize}%
    \setlength{\itemsep}{6pt}%
    \setlength{\parskip}{6pt}}%
  {\end{itemize}}

\newenvironment{enumeratex}%
  {\large \begin{enumerate}%
    \setlength{\itemsep}{6pt}%
    \setlength{\parskip}{6pt}}%
  {\end{enumerate}}


\title{Teorie informace a kódování (KMI/TIK)}
\subtitle{Reed-Mullerovy kódy}
\author{Lukáš Havrlant}
\date{4. prosince 2012}
\institute{Univerzita Palackého}

\begin{document}

\begin{frame}[t,plain]
\titlepage
\end{frame}


\begin{frame}[t,fragile]\frametitle{Primární zdroj} 
Jiří Adámek: Kódování. Strany 81--88.
\end{frame}


\begin{frame}[t,fragile]\frametitle{Boolovské funkce} 
    \begin{itemizex}
        \item Boolovská funkce $m$ proměnných je zobrazení $\cela_2^m\longrightarrow\cela_2$.
        \item Zapisujeme např. pomocí tabulky:
$$\begin{array}{l|cccc}
x_1&0&1&0&1\\
x_2&0&0&1&1\\\hline
f'&1&0&1&1
\end{array}
\qquad
\begin{array}{l|cccccccc}
x_1&0&1&0&1&0&1&0&1\\
x_2&0&0&1&1&0&0&1&1\\
x_3&0&0&0&0&1&1&1&1\\\hline
f''&1&1&0&1&1&1&0&1\\
\end{array}$$
        \item Hodnoty proměnných $x_i$ zapisujeme tak, aby sloupce tvořily čísla $0,1,\dots,2^m-1$ v binárním zápise.
    \end{itemizex}
\end{frame}


\begin{frame}[t,fragile]\frametitle{Reprezentace boolovské funkce} 

Boolovskou funkci o $m$ proměnných můžeme reprezentovat jako slovo délky $2^m$. Např. funkce
$$\begin{array}{l|cccc}
x_1&0&1&0&1\\
x_2&0&0&1&1\\\hline
f'&1&0&1&1
\end{array}
\qquad
\begin{array}{l|cccccccc}
x_1&0&1&0&1&0&1&0&1\\
x_2&0&0&1&1&0&0&1&1\\
x_3&0&0&0&0&1&1&1&1\\\hline
f''&1&1&0&1&1&1&0&1\\
\end{array}$$
        můžeme reprezentovat jako slova $f'=1011$ a $f''=11011101$, tj. jen jako poslední řádek. 

\end{frame}



\begin{frame}[t,fragile]\frametitle{Definice boolovské funkce} 
\begin{definice}
Slovo $f\in\cela_2^{2^m}$ tvaru $f=f_0f_1\dots f_{2^m-1}$, nazveme boolovskou funkcí o $m$ proměnných, definovanou předpisem $f(j_1, j_2, \dots, j_m)$ je rovno $f_j$, pokud má $j$ binární rozvoj $j_mj_{m-1}\dots j_1$.
\end{definice}

Příklad: nechť $f=11110010$ je boolovská funkce o 3 proměnných. Pak $f(0,0,1)$ vyhodnotíme jako  číslo $100_2$, což odpovídá číslu $4_{10}$, takže $f(0,0,1)=f_4=0$.

$$
\begin{array}{l|cccccccc}
x_1&0&1&0&1&0&1&0&1\\
x_2&0&0&1&1&0&0&1&1\\
x_3&0&0&0&0&1&1&1&1\\\hline
f&1&1&1&1&0&0&1&0
\end{array}
$$
\end{frame}


\begin{frame}[t,fragile]\frametitle{Zajímavé funkce} 
    \begin{itemizex}
        \item Nulová funkce $\zero=0\dots0$
        \item Funkce $\one=1\dots1$
        \item Samotné proměnné $x_i$ jsou také funkcemi. Např. pro $m=2$ máme $x_1=0101$ a $x_2=0011$.
        \item Ve funkci $x_i$ se vždy pravidelně střídají skupiny $2^{i-1}$ nul a $2^{i-1}$ jedniček.
    \end{itemizex}
\end{frame}


\begin{frame}[t,fragile]\frametitle{Logické operace} 
    \begin{itemizex}
        \item Použijeme definice operací $+$ a $\cdot$, které jsme použili u binárních lineárních kódů, tj. sčítání a násobení modulo 2.
        \item Pak pro boolovské funkce $f=\left<f_0,f_1,\dots,f_{2^m-1}\right>$ a $g=\left<g_0,g_1,\dots, g_{2^m-1}\right>$ a prvek $x\in\cela_2$ definujeme operace:
        \begin{eqnarray*}
f\cdot g&=& \left<f_0\cdot g_0, f_1\cdot g_1,\dots, f_{2^m-1}\cdot g_{2^m-1}\right>\\
f+ g&=& \left<f_0+ g_0, f_1+ g_1,\dots, f_{2^m-1}+ g_{2^m-1}\right>\\
x\cdot f=f\cdot x&=& \left<x\cdot f_0, x\cdot f_1,\dots, x\cdot f_{2^m-1}\right>\\
x+f=f+x&=& \left<x+f_0, x+f_1,\dots, x+f_{2^m-1}\right>\\
\neg f&=& 1+f
        \end{eqnarray*}
    \end{itemizex}
\end{frame}


\begin{frame}[t,fragile]\frametitle{Pozorování} 
    \begin{itemizex}
        \item Pokud máme funkci $f$ o třech proměnných, pak slovo $f_0f_1f_2f_3$ je zároveň funkce dvou proměnných tvaru $f(x_1, x_2, 0)$.
$$
\begin{array}{l|cccccccc}
x_1&0&1&0&1&0&1&0&1\\
x_2&0&0&1&1&0&0&1&1\\
x_3&0&0&0&0&1&1&1&1\\\hline
f&0&1&1&1&0&0&1&0
\end{array}
$$
        \item Pokud definujeme $g(x_1, x_2)=f(x_1, x_2, 0)$, pak $g=0111$, první čtyři symboly funkce $f$. Podobně pro $f(x_1, x_2, 1)$.
        \item Obecně: $f(x_1, x_2, \dots, x_{m-1}, 0)=f_0f_1\dots f_{2^{m-1}-1}$ a $f(x_1, x_2, \dots, x_{m-1}, 1)=f_{2^{m-1}}f_{2^{m-1}+1}\dots f_{2^{m}-1}$.
    \end{itemizex}
\end{frame}



\begin{frame}[t,fragile]\frametitle{Boolovské polynomy} 
    \begin{itemizex}
        \item Boolovské funkce můžeme také reprezentovat jako součty a součiny funkcí $x_i$ a $\one$.
        \item Příklad: $f=0110$. 
$$
\begin{array}{l|cccc}
x_1&0&1&0&1\\
x_2&0&0&1&1\\\hline
f&0&1&1&0
\end{array}
$$
        Funkci můžeme přepsat takto: $f(x_1, x_2)=x_1+x_2$.
        \item Tomuto tvaru říkáme boolovský polynom.
    \end{itemizex}
\end{frame}


\begin{frame}[t,fragile]\frametitle{Boolovské polynomy 2} 
  \begin{itemizex}
    \item Platí, že $x_i^2=x_i$. V polynomu se tak nevyskytují vyšší mocnitelé než 1.
    \item Boolovský polynom $m$ proměnných je součet funkce $\one$ a členů
    $$
    x_1^{i_1} x_2^{i_2}\dots x_m^{i_m},
    $$
    kde $i_1, \dots, i_m\in\cela_2$. Přitom $x_i^0=\one, x_i^1=x_i$. Pro $m=3$ máme:
    \begin{eqnarray*}
  x_1^1x_2^0x_3^0&=& x_1\\
  x_1^0x_2^1x_3^1&=& x_2x_3\\
  x_1^0x_2^0x_3^0&=& \one\\
    \end{eqnarray*}
  \end{itemizex}
\end{frame}


\begin{frame}[t,fragile]\frametitle{Boolovské polynomy -- definice} 
  \begin{definice}
  Boolovský polynom reprezentující boolovskou funkci $f$ o $m$ proměnných je výraz ve tvaru
  $$
  f(x_1, x_2, \dots, x_m) = \sum_{i=0}^{2^{m}-1}q_i\cdot x_1^{i_1}x_2^{i_2}\dots x_m^{i_m},
  $$
  kde $q_i\in\cela_2$ a číslo $i$ má binární rozvoj $i_mi_{m-1}\dots i_2i_1$.
  \end{definice}

  Pro funkce dvou proměnných tak má polynom tvar
  $$
  q_0\one+q_1x_1+q_2x_2+q_3x_1x_2.
  $$
\end{frame}


\begin{frame}[t,fragile]\frametitle{Příklad} 
    \begin{itemizex}
        \item Pro $f=x_2+x_1x_2$ máme:
$$
q_0+q_1x_1+q_2x_2+q_3x_1x_2,
$$
        kde $q_0=q_1=0,\quad q_2= q_3=1$.
        \item Pro $f=00001111$ máme:
$$
q_0+q_1x_1+q_2x_2+q_3x_1x_2+q_4x_3+q_5x_1x_3+q_6x_2x_3+q_7x_1x_2x_3,
$$
        kde $q_4=1$ a zbytek je nula. 
    \end{itemizex}
\end{frame}


\begin{frame}[t,fragile]\frametitle{Snížení počtu proměnných} 
    \begin{veta}
    Pro každou boolovskou funkci $m+1$ proměnných $f(x_1,\dots,x_m, x_{m+1})$ platí
    $$
f = f(x_1,\dots,x_m,0)+[f(x_1,\dots,x_m,0)+f(x_1, \dots, x_m,1)]\cdot x_{m+1}
    $$
    \end{veta}

    Důkaz: Dosadíme za $x_{m+1}$ jedna a nula. 
\end{frame}


\begin{frame}[t,fragile]\frametitle{Příklad} 
Převedeme funkci dvou proměnných $f=0111$ na polynom.
$$
f=01+(01+11)x_2=01+10x_2
$$
Dále vidíme, že 
\begin{eqnarray*}
01&=& 0+(0+1)x_1=x_1\\
10&=& 1+(1+0)x_1=1+x_1
\end{eqnarray*}
Takže 
$$
f=01+10x_2=x_1+(1+x_1)x_2=x_1+x_2+x_1x_2.
$$
\end{frame}


\begin{frame}[t,fragile]\frametitle{Množiny $M(i)$} 
    \begin{definice}
        Nechť má číslo $i$ binární rozvoj $i_{n}\dots i_2i_1$. Pak $M(i)$ je množina všech čísel $j\le i$ takových, že jejich binární rozvoj je $j_{n}\dots j_2j_1$ a přitom platí, že pokud $j_k=1$, pak $i_k=1$ pro všechna $k\in\{1,\dots,n\}$.
        $$
M(i_{n}\dots i_2i_1)=\{j_{n}\dots j_2j_1\sep \forall k \in \{1,\dots,n\}: j_k=1\Rightarrow i_k=1\}
        $$
    \end{definice}

    Příklady: 
    $$
    M(0)=\{0\},\quad M(1) = \{0,1\},\quad M(2)=\{0,2\},\quad M(3)=\{0,1,2,3\}.
    $$

    Jak budou vypadat množiny $M(2^i)$ a $M(2^i-1)$ pro $i\in\mathbb{N}$? 
\end{frame}


\begin{frame}[t,fragile]\frametitle{Množina $M(i+2^m)$} 
(Zde budeme výjimečně uvažovat klasický neobrácený binární rozvoj.) Nechť $i,m\in\mathbb{N}$ a $2^m\le i < 2^{m+1}$. Pak číslo $i$ má binární rozvoj $i=i_1i_2\dots i_{m+1}$, kde $i_1=1$. Např. pro $m=2$ máme čísla 
$$
4_{10}=100_2,\quad 5_{10}=101_2,\quad 6_{10}=110_2,\quad 7_{10}=111_2.
$$
Číslo $2^m$ pak má tvar $10\dots0$ ($m$ nul). Součet $i+2^m$ bude mít tvar: $10i_2\dots i_{m+1}$. Tj. čísla $i_2\dots i_{m+1}$ zůstanou v součtu stejná. Např. $5+4=9$ neboli $101+100=1001$. 
\emptyline
Je to podobné, jako když v desítkové soustavě přičítáme k číslu $i$ číslo $10^m$. Také se posledních $m$ číslic čísla $i$ nezmění. 
\emptyline
Jaký bude vztah množin $M(i)$ a $M(i+2^m)$?
\emptyline

\end{frame}


\begin{frame}[t,fragile]\frametitle{Pokračování\dots} 
Vidíme, že čísla $i$ a $i+2^m$ se liší pouze v prvních dvou symbolech. Tedy pro všechna $j\in M(i)$ s rozvojem $j=j_1j_2\dots j_{m+1}$, která mají $j_1=0$ platí, že $j\in M(i+2^m)$. Neboli pro všechna $j\in M(i)$ taková, že $j<2^m$, platí $j\in M(i+2^m)$.
\emptyline
Součet $i+2^m$ má tedy tvar $i_0i_1i_2\dots i_{m+1}$, kde $i_0=1, i_1=0$. Pokud v $M(i)$ bylo $j'$, kde $j'_1=1$, tak v $M(i+2^m)$ bude $j''$, kde $j''_0=1, j''_1=0$, přičemž ostatní číslice budou stejné, tj. $j'_k=j''_k$ pro $k\in\{2,\dots m+1\}$. Tedy pokud $j\in M(i)$ a $j\ge2^m$, pak $j+2^m\in M(i+2^m)$.
\emptyline
Můžeme tak napsat:
$$
M(i+2^m) = \{j\sep j\in M(i), j < 2^m \} \cup \{j+2^m\sep j\in M(i), j\ge 2^m\}.
$$
\end{frame}



\begin{frame}[t,fragile]\frametitle{Jak vyjádřit polynom z funkce} 
\begin{veta}
 Každá boolovská funkce $f=f_0\dots f_{2^m-1}$ je polynomem
    $$
f = \sum_{i=0}^{2^{m}-1}q_i\cdot x_1^{i_1}x_2^{i_2}\dots x_m^{i_m},
    $$
    kde $i$ má binární rozvoj $i_m\dots i_2i_1$ a koeficienty $q_i$ jsou rovny
    $$
    q_i=\sum_{j\in M(i)}f_j.
    $$
\end{veta}
\end{frame}


\begin{frame}[t,fragile]\frametitle{Příklad} 
    Vyjádříme funkci $f=1101$:
    \begin{eqnarray*}
    M(0)=\{0\},\quad&& q_0= f_0=1\\
    M(1)=\{0,1\},\quad&& q_1= f_0+f_1=0\\
    M(2)=\{0,2\},\quad&& q_2= f_0+f_2=1\\
    M(3)=\{0,1,2,3\},\quad&&q_3=f_0+f_1+f_2+f_3=1
    \end{eqnarray*}

    $$
f=q_0+q_1x_1+q_2x_2+q_3x_1x_2=1+x_2+x_1x_2.
    $$
\end{frame}


\begin{frame}[t,fragile]\frametitle{Důkaz} 
Máme dokázat:
$$
f = \sum_{i=0}^{2^{m}-1}q_i\cdot x_1^{i_1}x_2^{i_2}\dots x_m^{i_m},\quad q_i=\sum_{j\in M(i)}f_j.
$$
Postupujeme matematickou indukcí. Pro $m=1$ snadno ověříme:
$$
f(x_1) = q_01+q_1x_1=f_0+(f_0+f_1)x_1.
$$

\end{frame}


\begin{frame}[t,fragile]\frametitle{Důkaz: pokračování} 
\footnotesize
Použijeme indukční předpoklad na funkci $f(x_1,\dots,x_m,0)$, která má $m$ proměnných:
$$
f(x_1,\dots,x_m,0)=\sum_{i=0}^{2^m-1}q_i'\cdot x_1^{i_1}\dots x_m^{i_m}.
$$
Koeficienty $q_i'$ se rovnají (Využíváme toho, že $x_{m+1}$ má v levé polovině pravdivostní tabulky hodnotu 0, v pravé polovině 1. Dále využíváme toho, že pokud $m$-ciferný rozvoj čísla $j$ je $j_m\dots j_1$, pak jeho $(m+1)$-ciferný rozvoj je $0j_m\dots j_1$):
$$
q_i'=\sum_{j\in M(i)} f(j_1,\dots,j_m,0) =\sum_{j\in M(i)}f_j=q_i.
$$
Totéž pro funkci $f(x_1,\dots,x_m,1)$:
$$
f(x_1,\dots,x_m,1)=\sum_{i=0}^{2^m-1}q_i''\cdot x_1^{i_1}\dots x_m^{i_m},\qquad q_i''=\sum_{j\in M(i)} f(j_1,\dots,j_m,1) =\sum_{j\in M(i)}f_{j+2^m}.
$$
\end{frame}


\begin{frame}[t,fragile]\frametitle{Důkaz: pokračování\dots} 
\footnotesize
Nyní využijeme vzorce 
$$
f(x_1,\dots,x_m, x_{m+1}) = f(x_1,\dots,x_m,0)+[f(x_1,\dots, x_m,0)+f(x_1, \dots, x_m,1)]\cdot x_{m+1}
$$
a rozepíšeme naši funkci $f(x_1,\dots,x_m, x_{m+1})$ pomocí předchozích výsledků:

\begin{eqnarray*}
f(x_1,\dots,x_m, x_{m+1}) &=&  \sum_{i=0}^{2^m-1}q_i x_1^{i_1}\dots x_m^{i_m}+\left[\sum_{i=0}^{2^m-1}q_i x_1^{i_1}\dots x_m^{i_m}+\sum_{i=0}^{2^m-1}q_i'' x_1^{i_1}\dots x_m^{i_m}\right]x_{m+1}\\
&=& \sum_{i=0}^{2^m-1}q_i x_1^{i_1}\dots x_m^{i_m}+\sum_{i=0}^{2^m-1}(q_i+q_i'') x_1^{i_1}\dots x_m^{i_m}x_{m+1}.
\end{eqnarray*}

Tento výraz obsahuje všechny součiny $x_1\dots x_mx_{m+1}$. Ty součiny, které neobsahují $x_{m+1}$ jsou v levé sumě, ty, které obsahují $x_{m+1}$ jsou v pravé sumě. Koeficienty u součinů jsou $q_i$ a $(q_i+q_i'')$. Nyní tak musíme dokázat, že tyto koeficienty jsou stejné, jako koeficienty u původní funkce $f(x_1,\dots,x_m, x_{m+1})$.


\end{frame}


\begin{frame}[t,fragile]\frametitle{Důkaz: finále} 
\footnotesize
% $$
% \sum_{i=0}^{2^m-1}q_i x_1^{i_1}\dots x_m^{i_m}+\sum_{i=0}^{2^m-1}(q_i+q_i'') x_1^{i_1}\dots x_m^{i_m}x_{m+1}
% $$
Všechna čísla $0,\dots,2^{m+1}$ lze rozdělit na: $\le2^m-1$ s binárním rozvojem $0i_m\dots i_1$. Pak platí, že koeficient u členu
$$
x_i^{i_1}\dots x_m^{i_m}=x_i^{i_1}\dots x_m^{i_m}x_{m+1}^{0}
$$
je $q_i$. Dále pro $\ge2^m$ s rozvojem $1i_m\dots i_1$ platí, že koeficient u členu $x_i^{i_1}\dots x_m^{i_m}x_{m+1}^1$ je rovný $q_i+q_i''$. Dále dokážeme, že $q_{i+2^m}=q_i+q_i''$. Platí, že
$$
q_i''=\sum_{j\in M(i)} f(j_1,\dots,j_m,1) =\sum_{j\in M(i)}f_{j+2^m},\qquad q_i=\sum_{j\in M(i)} f(j_1,\dots,j_m,0)=\sum_{j\in M(i)}f_j
$$
Tedy pokud sečteme $q_i+q_i''$, dostaneme:
$$
q_i+q_i''=\sum_{j\in M(i)}f_j+\sum_{j\in M(i)}f_{j+2^m}=\sum_{j\in M(i+2^m)}f_j=q_{i+2^m}.
$$
Zde jsme využili faktu, že $M(i+2^m)$ obsahuje všechna $j\le 2^m-1$, která leží v $M(i)$ a dále obsahuje všechna $j+2^m$ pro $j\in M(i)$.\qed
\end{frame}


\begin{frame}[t,fragile]\frametitle{Báze lineárního prostoru $\cela_2^{2^m}$} 
    Z předchozí věty víme, že každá funkce / každé slovo lze vyjádřit pomocí součtů součinů funkcí $\one, x_1,\dots, x_m$. Funkce
    \begin{eqnarray*}
    &&\one,\\
    &&x_1,x_2,\dots,x_m,\\
    &&x_ix_j\quad\mbox{pro}\quad i,j\in\{1,\dots,m\},\\
    &&\vdots,\\
    &&x_1x_2\dots x_m
    \end{eqnarray*}
    tak tvoří bázi lineárního prostoru $\cela_2^{2^m}$. Nezávislost funkcí plyne z počtu těchto funkcí. Počet funkcí v bázi je roven:
$$
{m\choose0}+{m\choose1}+{m\choose2}+\dots+{m\choose m-1}+{m\choose m}.
$$

\end{frame}

\begin{frame}[t,fragile]\frametitle{Nezávislost} 
Přitom binomická věta říká:
    $$
(x+y)^m=\sum_{k=0}^{m}{m\choose k}x^{m-k}y^k.
    $$
Pokud dosadíme $x=y=1$, máme:
$$
2^m=\sum_{k=0}^{m}{m\choose k}={m\choose0}+{m\choose1}+{m\choose2}+\dots+{m\choose m-1}+{m\choose m}.
$$
Počet funkcí v bázi je tak roven $2^m$, což je dimenze prostoru $\cela_2^{2^m}$.
\end{frame}


\begin{frame}[t,fragile]\frametitle{Stupeň polynomu} 
    Stupeň boolovského polynomu
        $$
        f = \sum_{i=0}^{2^{m}-1}q_i\cdot x_1^{i_1}x_2^{i_2}\dots x_m^{i_m}
        $$
        je maximální váha slova $i_1i_2\dots i_m$ takového, že $q_i=1$. Tedy je to největší počet proměnných, které se vyskytují v nějakém sčítanci. 
\emptyline
        Příklady:
        \begin{itemizex}
            \item Polynom $1+x_1+x_1x_2$ má stupeň 2,
            \item polynom $x_1x_3+x_1x_2x_3$ má stupeň 3.
        \end{itemizex}
\end{frame}


\begin{frame}[t,fragile]\frametitle{Reed-Mullerovy kódy}
    \begin{itemizex}
        \item Binární kód délky $n=2^m$ pro $m\in\mathbb{N}$.
        \item Kódová slova jsou boolovské polynomy $m$ proměnných.
    \end{itemizex}

    \begin{definice}
    Reed-Mullerovým kódem stupně $r$ a délky $2^m$ se nazývá množina $R(r,m)$ všech boolovských polynomů $m$ proměnných stupně nejvýše $r$.
    \end{definice}
\end{frame}


\begin{frame}[t,fragile]\frametitle{Příklad Reed-Mullerových kódů} 
\begin{figure}[htb]
    \centering
    \includegraphics[scale=0.7]{img/reed.pdf}
\end{figure}
\end{frame}



\begin{frame}[t,fragile]\frametitle{Generující matice} 
    \begin{itemizex}
        \item Reed-Mullerův kód je lineární, protože součet dvou polynomů stupně nejvýše $r$ je také polynom stupně nejvýše $r$.
        \item Generující matice tvoří všechny polynomy obsahující jediný výraz ve tvaru součinu o stupni $\le r$. Matice kódu $R(2,3)$:
    \end{itemizex}

\begin{figure}[htb]
    \centering
    \includegraphics[scale=1]{img/reed-matice.pdf}
\end{figure}
\end{frame}


\begin{frame}[t,fragile]\frametitle{Kódování} 
    \begin{itemizex}
        \item Slovo, které chceme zakódovat, má délku $k$, kde
        $$
k=\sum_{i=0}^r{m\choose i}
        $$
        \item Pokud chceme zakódovat slovo $u$, pak symboly $u_i$ nám říkají, které součiny z generující matice budou ve výsledném polynomu.
        \item Příklad: chceme zakódovat slovo $0010110$. Použijeme předchozí matici $G$. Pak 
        $$
0010110\cdot G=x_2+x_1x_2+x_1x_3=00100111     
        $$
    \end{itemizex}
\end{frame}


\begin{frame}[t,fragile]\frametitle{Vlastnosti} 
    \begin{itemizex}
        \item $R(m,m)=\cela_2^{2^m}$
        \item $R(m-1,m)$ je $(2^m,2^m-1)$ kód sudé parity (ale není systematický, viz příklad).
        \item $R(0, m)$ je $(2^m,1)$ opakovací kód. 
        \item Minimální vzdálenost $R(r,m)$ kódu je $2^{m-r}$.
        \item $R(1,5)$ je $(32,6)$ kód, který využil kosmický koráb Mariner 9 při vysílání fotografií z Marsu v roce 1972.
    \end{itemizex}
\end{frame}



\end{document}